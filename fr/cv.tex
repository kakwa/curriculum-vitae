% start of file `template_en.tex'.
% Copyright 2006-1008 Xavier Danaux (xdanaux@gmail.com).
%
% This work may be distributed and/or modified under the
% conditions of the LaTeX Project Public License version 1.3c,
% available at http://www.latex-project.org/lppl/.

\documentclass[10pt,a4paper]{moderncv}
\usepackage[utf8x]{luainputenc}
\usepackage[T1]{fontenc}

\linespread{0.9}

% moderncv themes
% optional argument are 'blue' (default), 'orange', 'red', 'green', 'grey' and 'roman'
% (for roman fonts, instead of sans serif fonts)
\moderncvtheme[blue]{classic}                 
%\usepackage{fontawesome}

\usepackage{moderncviconsawesome}
%\moderncvicons{awesome}
%\usepackage[latin1]{inputenc}

% character encoding
% replace by the encoding you are using


% adjust the page margins
\usepackage[scale=0.9]{geometry}

% if you want to change the width of the column with the dates
%\setlength{\hintscolumnwidth}{1cm}  

% only for the classic theme, if you want to change the width of
% your name placeholder (to leave more space for your address details
\AtBeginDocument{\setlength{\maketitlenamewidth}{8cm}}

% required when changes are made to page layout lengths
\AtBeginDocument{\recomputelengths}

% personal data
\firstname{Pierre-François}
\familyname{Carpentier}

% optional, remove the line if not wanted
%title{}

% optional, remove the line if not wanted
\address{41 rue de la Tombe Issoire}{75014 Paris}    

% optional, remove the line if not wanted
\mobile{06 70 47 10 17}

% optional, remove the line if not wanted
%\phone{phone (optional)}

% optional, remove the line if not wanted
%\fax{fax (optional)}

% optional, remove the line if not wanted
\email{carpentier.pf@gmail.com}

% optional, remove the line if not wanted
% \extrainfo{Permis B} 
\extrainfo{%
\faGithub~\httplink{www.github.com/kakwa/}\\%
\faLinkedinSign~\httplink{fr.linkedin.com/in/pfcarpentier/}\\
Permis B
}
%
% '64pt' is the height the picture must be resized to 
% and 'picture' is the name of the picture file; 
% optional, remove the line if not wanted
\photo[64pt]{ID.jpg}

% uncomment to suppress automatic page numbering for CVs longer than one page
%\nopagenumbers{}


%----------------------------------------------------------------------------------
%            content
%----------------------------------------------------------------------------------
\begin{document}
\maketitle

\section{Expérience}
\subsection{Stages}

\cventry{2010}{Developpement web: création et déploiement d'un site web}{INRIA}{Roquencourt}{}{Stage dans le cadre du projet européen Ideas, 
conception, réalisation et deploiement d'un site permettant de visionner des manuscrits orientaux ainsi que leurs traductions.
 Le site a été réalisé grace au framework Django. Il a par ailleurs fallu mettre en place un serveur pour heberger le site.
\newline Technologies utillisées: Python, Django, HTML, CSS, Apache, Linux.}                % arguments 3 to 6 are optional

\cventry{2009}{Modélisation biologique: création d'un modèle de tumeur}{Laboratoire IBISC}{EVRY}{}{Stage de recherche en Bio-Informatique. Conception
et réalisation d'un modèle de tumeur
cancéreuse afin d'étudier l'action de la molécule PA-1 (molécule suspectée de favoriser la migration des cellules cancéreuses et donc l'apparition de
métastases). \newline Technologies utillisées: C, Doxygen.}                % arguments 3 to 6 are optional


\subsection{Divers}
\cventry{2008}{Stagiaire au Crédit Mutuel de Bretagne}{}{Brest}{}{Stage d'un mois, au département du Libre Service Bancaire (gestion des automates bancaires).
Mise à jour et amélioration d'une étude de fréquentation des distributeurs suite à l'installation d'une nouvelle interface
utilisateur. Tests pratiques sur la nouvelle interface avant sa mise en production.
}

\section{Langues}
\cvlanguage{Anglais}{Courant}{Score de 940/990 au TOEIC, Pratique quotidienne}
\cvlanguage{Espanol}{Notions}{Niveau Bac}

\section{Compétences}
\cvline{OS}{Gnu/Linux, Mac OSX, Ms Windows}{}{}
\cvline{Scripting}{Bash, Python}{}{}
\cvline{Réseau}{Sécurité, Routage, QoS, TCP/IP, ATM, MPLS}{}{}
%\cvline(Réseau}{Sécurité, Routage, QoS, TCP/IP, ATM, MPLS}{}{}
\cvline{Théorie}{Recherche Opérationnelle, Compilation, Théorie des graphes, Analyse de Données, UML}{}{}
\cvline{Programmation}{C, Java, Ocaml, Prolog, VHDL, ASM, Coq}{}{}
\cvline{Développement web}{Django, HTML, PHP,CSS, PostgreSQL}{}{}
\cvline{Divers}{Open Office, Ms Office, Lex \& Yacc, RPC, Corba, \LaTeX}{}{}

\section{Formation}
\cventry{2010--2011}{Master Recherche Réseau}
{Université Pierre et Marie Curie Paris 6}{Paris}{}{Master effectué en parallèle avec la 3ème année d’école d’ingénieur. Sujets : Sécurité, Routage,
QoS, Réseaux, P2P. . .}

\cventry{2008--2011}{Etudes d'ingénieur en Informatique}{Ensiie}{Evry}{}
{Etudes à l'Ecole Nationale Supérieure d'Informatique pour l'Industrie et l'Entreprise (ENSIIE), grande école d'ingénieurs
généraliste en informatique recrutant sur le concours Centrale-Supelec. Spécialisation Système et Réseaux.
}  % arguments 3 to 6 are optional
\cventry{2005--2008}{Classe Préparatoire aux Grandes Ecoles}
{Lycée de Kerichen}{Brest}{}{Filière MP (Mathématiques-Physique) option Informatique.}



\section{Centres d'intéret}
\cvline{Sport}{\small Pratique de la voile en compétition pendant neuf ans.}
\cvline{Musique}{\small Apprentissage et pratique de la guitare et du piano depuis plusieurs années, ainsi que, depuis peu, de la batterie.}
\cvline{Electronique}{\small Fabrication et réparation d'appareils divers et variés.}
\cvline{Informatique}{\small Parallèlement à ma formation, je me renseigne sur tout ce qui touche à l'informatique (de l'histoire de l'informatique à
l'architecture des systèmes d'exploitation en passant par l'apprentissage du langage Python.)
}

\end{document}

   
%% end of file `template_en.tex'.
