%% start of file `template.tex'.
%% Copyright 2006-2015 Xavier Danaux (xdanaux@gmail.com).
%
% This work may be distributed and/or modified under the
% conditions of the LaTeX Project Public License version 1.3c,
% available at http://www.latex-project.org/lppl/.


\documentclass[10pt,a4paper,sans]{moderncv}        % possible options include font size ('10pt', '11pt' and '12pt'), paper size ('a4paper', 'letterpaper', 'a5paper', 'legalpaper', 'executivepaper' and 'landscape') and font family ('sans' and 'roman')
\usepackage{ifxetex}

\ifxetex
  \usepackage{fontspec}
\else
  \usepackage[T1]{fontenc}
  \usepackage[utf8]{inputenc}
  \usepackage{lmodern}
\fi

% moderncv themes
\moderncvstyle{classic}                             % style options are 'casual' (default), 'classic', 'banking', 'oldstyle' and 'fancy'
\moderncvcolor{blue}                               % color options 'black', 'blue' (default), 'burgundy', 'green', 'grey', 'orange', 'purple' and 'red'
%\renewcommand{\familydefault}{\sfdefault}         % to set the default font; use '\sfdefault' for the default sans serif font, '\rmdefault' for the default roman one, or any tex font name
%\nopagenumbers{}                                  % uncomment to suppress automatic page numbering for CVs longer than one page

% character encoding
%\usepackage[utf8]{inputenc}                       % if you are not using xelatex ou lualatex, replace by the encoding you are using
%\usepackage{CJKutf8}                              % if you need to use CJK to typeset your resume in Chinese, Japanese or Korean

% adjust the page margins
\linespread{0.9}
\usepackage[scale=0.9]{geometry}
%\setlength{\hintscolumnwidth}{3cm}                % if you want to change the width of the column with the dates
%\setlength{\makecvheadnamewidth}{10cm}            % for the 'classic' style, if you want to force the width allocated to your name and avoid line breaks. be careful though, the length is normally calculated to avoid any overlap with your personal info; use this at your own typographical risks...

% personal data
\name{Pierre-Francois}{Carpentier}
%\title{Resumé title}                               % optional, remove / comment the line if not wanted
\address{41 rue de la Tombe Issoire}{75014 Paris}

\phone[mobile]{+33 (0) 670 471 017}

\email{carpentier.pf@gmail.com}

\social[linkedin]{pfcarpentier}                        % optional, remove / comment the line if not wanted
%\social[twitter]{jdoe}                             % optional, remove / comment the line if not wanted
\social[github]{kakwa}                              % optional, remove / comment the line if not wanted
%\extrainfo{additional information}                 % optional, remove / comment the line if not wanted
\extrainfo{
Permis B
}
% '64pt' is the height the picture must be resized to 
% and 'picture' is the name of the picture file; 
% optional, remove the line if not wanted
\photo[64pt]{ID.jpg}

% uncomment to suppress automatic page numbering for CVs longer than one page
%\nopagenumbers{}


%----------------------------------------------------------------------------------
%            content
%----------------------------------------------------------------------------------
\begin{document}
\maketitle

\section{Compétences}

\subsection{Système}
\cvcomputer{OS}{Debian, CentOS, Gentoo, FreeBSD}{Virtualisation}{ESXi, VirtualBox, Vagrant, KVM}
\cvcomputer{Packaging}{deb, rpm, ebuild}{Déploiement}{Puppet, Kickstart, Preseed}
\cvcomputer{Réseau}{Keepalived, OpenVPN, DHCPd, VLANs}{}{}

\subsection{Services}
\cvcomputer{Serveurs Web}{Apache, Lighttpd, Nginx, Tomcat}{Annuaires}{389 Directory, OpenLDAP, Samba4}
\cvcomputer{Base de données}{MySQL, PosgreSQL, Redis}{Autres services}{Bind, Ntpd, Cups}
\cvcomputer{Supervision}{Nagios, NRPE, Logstash, Snmpd}{Sécurité}{Syslog-ng, Rsyslog, OpenSSH, IPTables, PF}
\cvcomputer{Network FS}{Samba, NFS}{}{}

\subsection{Programmation}
\cvcomputer{Langages}{Shell/Bash, Python, Perl, C, Ocaml, Java}{Gestionnaires de révisions}{Git, Subversion, Mercurial}
\cvcomputer{Build}{Make, CMake, gcc, clang}{}{}

\subsection{Autre}
\cvcomputer{Théorie des Réseaux}{Sécurité, Routage, QoS, TCP/IP, ATM, MPLS}{Théorie Informatique}{Recherche Opérationnelle, Compilation, Théorie des graphes, Analyse de Données, UML}
\cvcomputer{Documentation}{reStructuredText, Markdown, \LaTeX, OpenOffice}{}{}

\section{Langues}
\cvlanguage{Anglais}{Courant}{Score de 940/990 au TOEIC, Pratique quotidienne}
\cvlanguage{Espagnol}{Notions}{Niveau Bac}


\section{Expérience}
\subsection{Parcours professionnel}

\cventry{2011--présent}{Ingénieur Intégrateur}
              {Communication et Systèmes}
              {Le Plessis Robinson}
              {}
              {Réalisation de systèmes complexes pour clients Étatiques/Parapubliques:
                \begin{itemize}
                \item Mise en place de déploiement Automatique (Puppet)
                \item Packaging logiciel Debian et Redhat/CentOS
                \item Création de dépôts et signature de packages (Reprepro/Createrepo, GPG)
                \item Rédaction de scripts de build (Make, Ant)
                \item Automatisation et versionnement de la génération des livrables.
                \item Mise en place de mécanismes d'installation automatique (Kickstart/Preseed)
                \item Configuration DNS (Bind)
                \item Configuration SGBD (MySQL/PostgreSQL)
                \item Configuration annuaires (389 Directory/OpenLDAP)
                \item Configuration serveurs web (Apache, Nginx, Tomcat)
                \item Mise en place de mécanismes de haute disponibilité (Réplication SQL/LDAP, Keepalived)
                \item Configuration Debian/CentOS
                \item Mise en place de supervision (Nagios, NRPE, SNMP, Logstash)
                \item Virtualisation (VMware ESXi, VirtualBox/Vagrant, KVM)
                \item Scripting varié en Python/Shell/Perl/Ruby
                \item Rédaction de manuels et procédures d'exploitation (reStructuredText)
                \item Rédaction de fiches de tests (Testlink)
                \item Déploiement de solution de gestion de projet (Trac)
                \end{itemize}
              }

\subsection{Projet personnels}
\cventry{2012--present}{Divers projets open source publiés sur GitHub}
              {}
              {}
              {}
              {
              \begin{itemize}
              \item \href{https://github.com/kakwa/puppet-samba}{puppet-samba}, module Puppet pour gérer samba.
              \item \href{https://github.com/kakwa/py-ascii-graph}{ascii-graph}, librairie Python pour affciher des données sous forme d'histogrammes ascii.
              \item \href{https://github.com/kakwa/ldapcherry}{ldapcherry}, application web en Python/CherryPY pour gérer des utilisateurs/groupes dans de multiples annuaires/bases.
              \item \href{https://github.com/kakwa/libemf2svg}{libemf2svg}, librairie/utilitaire pour convertir des fichiers Microsoft (MS) EMF en SVG, écrit en C.
              \item \href{https://github.com/kakwa/libvisio2svg}{libvisio2svg}, librairie/utilitaires pour des fichiers Microsoft (MS) Visio (VSS et VSD) en SVG, écrit en C++.
              \item \href{https://github.com/kakwa/packages}{packages}, divers packages deb/rpm pour mes besoins.
              \item Beaucoup d'autres projets sur GitHub: \url{https://github.com/kakwa?tab=repositories}
              \end{itemize}
}

\subsection{Stages}

\cventry{2011}{Stage, Intégration/Développement: conception, d'une infrastructure de messagerie sécurisée}
              {Communication et Systèmes}
              {Le Plessis Robinson}
              {}
              {Conception et réalisation d'une infrastructure de messagerie chiffrée de bout en bout par
               Certificat X509. Conception et réalisation d'un POC de mailing list chiffrée de bout en bout
               (cryptographie à N destinataires).
               \newline Technologies utilisées: Linux, S/MIME, x509, thunderbird, OpenLDAP, TLS, C, librairie polarSSL.
              }

\cventry{2010}{Stage, Développement web: conception, réalisation et déploiement d'un site web}
              {INRIA}
              {Roquencourt}
              {}
              {Stage dans le cadre du projet européen Ideas, 
                 conception, réalisation et déploiement d'un site permettant de visionner 
                 des manuscrits orientaux ainsi que leurs traductions.
                 \newline Technologies utilisées: Python, Django, HTML, CSS, Apache, Linux.
              }

\cventry{2009}{Stage, Modélisation biologique: création d'un modèle de tumeur}
              {Laboratoire IBISC}
              {EVRY}
              {}
              {Stage de recherche en Bio-Informatique. Conception
                et réalisation d'un modèle de tumeur
                cancéreuse afin d'étudier l'action de la molécule PA-1.
                \newline Technologies utilisées: C, Doxygen.
              }

\cventry{2008}{Stage, Crédit Mutuel de Bretagne}{}{Brest}{}{Stage d'un mois, 
    au département du Libre Service Bancaire (gestion des automates bancaires).
    Mise à jour et amélioration d'une étude de fréquentation des distributeurs 
    suite à l'installation d'une nouvelle interface
    utilisateur. Tests pratiques sur la nouvelle interface avant sa mise en production.
}

\section{Formation}
\cventry{2010--2011}{Master Recherche Réseau}
{Université Pierre et Marie Curie Paris 6}{Paris}{}{Master effectué en parallèle avec la 3ème année d’école d’ingénieur. Sujets : Sécurité, Routage,
QoS, Réseaux, P2P. . .}

\cventry{2008--2011}{Études d'ingénieur en Informatique}{Ensiie}{Evry}{}
{Études à l'École Nationale Supérieure d'Informatique pour l'Industrie et l'Entreprise (ENSIIE), grande école d'ingénieurs
généraliste en informatique recrutant sur le concours Centrale-Supelec. Spécialisation Système et Réseaux.
} 
\cventry{2005--2008}{Classe Préparatoire aux Grandes Écoles}
{Lycée de Kerichen}{Brest}{}{Filière MP (Mathématiques-Physique) option Informatique.}

\section{Centres d'intérêt}
\cvline{Sport}{\small Pratique de la voile en compétition pendant neuf ans.}
\cvline{Musique}{\small Apprentissage et pratique de la guitare et du piano depuis plusieurs années.}
\cvline{Électronique}{\small Fabrication et réparation d'appareils divers et variés.}
\cvline{Informatique}{\small Plusieurs projets personnels (plusieurs librairies publiées sur GitHub), mis en place de services (Bind, OpenVPN, Lighttpd, NFS) pour mes propres besoins.
}

\end{document}

%% end of file `template_en.tex'.
